% LaTeX test file for kate's syntax highlighting and code folding

\ordinaryLaTeXcommandwithoption[10pt,a4paper]{article}
% BEGIN region
%comment, this is all ok % $ 
\%no comments please
\\%comment
% END of region

\newcommand{\nohighlighting}

\section{normal}

\ref{blue}
\pageref{blue}
\cite{blue}

\begin{environmentshavespecialcolors}
normal
\end{environmentshavespecialcolors}

$equations are green, \commands somewhat darker$
normal
$$equations are green, \commands somewhat darker$$
normal
\( 
\frac{1}{2}
\begin{test}
\end{test}
\)
normal
\[
%comment 
displaymath 
\]
normal
\begin{equation}
green\darkergreen
\begin{test}
\test
\end{test}
\end{equation}

\begin{equation*}
green\darkergreen
%comment
\begin{test}
\test
\end{test}
\%no comment
\end{equation*}

\{     %this should be comment

\verb%this shouldn't be%and this should be normal text
\verb!verbatim text! normal text

text \texttt{more text}

\begin{verbatim}
text inside a verbatim environment is also treated special $ %,
you can even put a \begin{verbatim} inside
\end{verbatim}

normal

\begin{Verbatim}
&@@#^%&^#$
\end{Verbatim}

\begin{Verbatim*}
@*&^#@*(^#(*@&
\end{Verbatim*}

normal

\begin{Verbatim}
\begin{verbatim}
This is a verbatim block.
\end{verbatim}
\end{Verbatim}

normal

% test alignat
\begin{alignat}{2}
A  &= B &= C \\
A  &= B &= C
\end{alignat}
normal text

\iffalse
\fill commented out text
\fi

% Math mode

Depending on the value of $x$ the equation \( f(x) = \sum_{i=0}^{n} \frac{a_i}{1+x} \) may diverge or converge.
 
\[ f(x) = \sum_{i=0}^{n} \frac{a_i}{1+x} \]
 
\[
S = \{ z \in \mathbb{C}\, |\, |z| < 1 \} \quad \textrm{and} \quad S_2=\partial{S}
\]

\[
\frac{
    \begin{array}[b]{r}
      \left( x_1 x_2 \right)\\
      \times \left( x'_1 x'_2 \right)
    \end{array}
  }{
    \left( y_1y_2y_3y_4 \right)
  }
\]

\begin{eqnarray*}
\begin{eqnarray*}
f(x) = \sum_{i=0}^{n} \frac{a_i}{1+x} \\
\textstyle f(x) = \textstyle \sum_{i=0}^{n} \frac{a_i}{1+x} \\
\scriptstyle f(x) = \scriptstyle \sum_{i=0}^{n} \frac{a_i}{1+x} \\
\scriptscriptstyle f(x) = \scriptscriptstyle \sum_{i=0}^{n} \frac{a_i}{1+x}
\end{eqnarray*}
\end{eqnarray*}

\begin{xalignat}{3}
i_{11} & =i_{23}\nonumber
\end{xalignat}

c
\begin{equation}
c
\begin{aligned}
a & b\\
c & d
\end{aligned}
c
\end{equation}
c

$$E=mc^2$$
 
\begin{equation}
E=m
\end{equation}

\begin{equation
x=3\textrm{plop}
\end{equation}

\[ \begin{array}{llll}
  x^3 = (-x)^3 & \text{if $x > 0$}\\
  x^3 = (-x)^3 & \text{if {$x > 0$}}\\
  x^3 = (-x)^3 & \text{if {\color{green} $x > 0$}}\\
\end{array} \]

\section*{Notes for My Paper}

\begin{center}
\begin{tabular}{ |c|c|c| } 
 \hline
 cell1 & cell2 & cell3 \\ 
 cell4 & cell5 & cell6 \\ 
 cell7 & cell8 & cell9 \\ 
 \hline
\end{tabular}

\begin{tabular*}{0.75\textwidth}{@{\extracolsep{\fill} } | c | c | c | r | }
  \hline
  label 1 & label 2 & label 3 & label 4 \\
  \hline 
  item 1  & item 2  & item 3  & item 4  \\
  \hline
\end{tabular*}

\begin{tabular*}{0.75\textwidth}{ | c | c | c | r | }
  \hline
  label 1 & label 2 & label 3 & label 4 \\
  \hline 
  item 1  & item 2  & item 3  & item 4  \\
  \hline
\end{tabular*}
\end{center}


\begin{tabularx}{1\textwidth}{ |>{\setlength\hsize{1\hsize}\centering}X|>{\setlength\hsize{1\hsize}\raggedleft}X@{} >{\setlength\hsize{1\hsize}\raggedright}X|>{\setlength\hsize{1\hsize}\centering}X| } 
  \hline
Label 1 & \multicolumn{2}{>{\centering\setlength\hsize{2\hsize} }X|}{Label 2} & Label 3\tabularnewline
\hline 
  123  & 123  & 456  & 123  \tabularnewline
  \hline
  123  & 123  & 456  & 123  \tabularnewline
  \hline
\end{tabularx}

\begin{lstlisting}
Write('Case insensitive ');
Write('Pascal keywords.');
\end{lstlisting}

\begin{lstlisting}%[frame=single]
Write('Case insensitive ');
Write('Pascal keywords.');
\end{lstlisting}

\begin{lstlisting}[frame=single]
Write('Case insensitive ');
Write('Pascal keywords.');
\end{lstlisting}

\begin{lstlisting}[frame=single] % blah blah
Write('Case insensitive ');
Write('Pascal keywords.');
\end{lstlisting}

\begin{lstlisting}
[frame=single] % blah blah
Write('Case insensitive ');
Write('Pascal keywords.');
\end{lstlisting}

\begin{minted}{python}
def foo(x):
    return x
\end{minted}

\begin{minted}
% blah blah
{python}
def foo(x):
    return x
\end{minted}

\begin{minted}[frame=lines]{python}
def foo(x):
    return x
\end{minted}

\begin{minted}
% plop
[frame=lines,
fontsize=\footnotesize
]
% ok
{python}
def foo(x):
    return x
\end{minted}

\begin{comment}
This is a comment block.
\end{comment}

\documentclass{article}
\usepackage{fancyvrb}

\documentclass[12pt]{article}
\begin{document}

Text that has a footnote\footnote{This $i_s$ the \[i_s\] $$x_i$$ \(x_i\)footnote}\addtocounter{footnote}{-1}\addtocounter{footnote}{-1} looks like this. Later text referring to same footnote\footnotemark uses the other command.

\end{document}

% A delimiter is not required before "\" in keywords of kind "\command" (see bug #413493)
\begin{equation}123\end{equation}

\begin{comment}Comment\end{comment}
\iffalse text\fi normal text

text\begin{tabular}text\end{tabular}
text\begin{a}text\end{a}

% Close brackets correctly in some commands (see bug #415384)
\input{\a{}}
\cites{ text { text } text }

% Parentheses within [ ... ] (see bug #418979)
\cite[eq. (1.1)]{some_reference}
\cite[eq. \(x^n + y^n = z^n\)]{some reference}

% Close folding correctly in "\end{...}" and in region markers (see bug #419125)
\begin{document}
  \begin{abstract}

  \end{abstract}

  \begin{itemize}
    \item[\(\bullet\)]
    %%  BEGIN something
    Here are some text describiing a new theorem. In the equation
    \begin{equation}
      E = mc^2
    \end{equation}
    we have a solution for energy-mass equivalence.
    %% END

    \item[\(\bullet\)]
    \begin{align*}
      B(z) &= p^r \sum_{k \geq 0} \binom{-r}{k} (-1)^k (1 - p)^k z^k \\
      B(z) &= p^r \sum_{k \geq 0} \binom{-r}{k} (-(1-p) z)^k
    \end{align*}
  \end{itemize}
\end{document}

% Allow Math environment within another Math environment (see bug #435086)
\begin{gather*}
    a = 5 \\
    y = \begin{cases}
      0 & x < 2 \\
      2x-4 & x > 2 \\
    \end{cases} \\
    x = 5 \\
    z = 5
\end{gather*}

\begin{bashcode}
    find . -name '*.jpg' -print | parallel --bar convert '{}' '{.}.png'
    \(ToRemember\) # Remember expression in brackets and then use
    "${RAXML}" -s ${INPUT} # All the parameters as usually...
\end{bashcode}

\[\begin{matrix}2\\3\\3\end{matrix}\]

\section{title}{\small bla bla}

\section  {title}
{\small bla bla}

\section
  {title}
{\small bla bla}

$\begin{tabu*} to 140mm {|X[$c]}
    \alpha & \beta & \gamma \\
    \sum_i \frac{a_i}{x_i} & 0 & \cdot \\
\end{tabu}$

\begin{tabu}{|c|*{21}{X[$c]|}}
    \toprule
    Notes $x_i$ & 0 & 0b\\
    \hline
    Notes $x_i$ & 0 & 0b\\
    \bottomrule
\end{tabu}
